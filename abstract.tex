\section*{Abstract}

\subsection*{Background}
Agencies such as the Centers for Disease Control (CDC) currently release incidence
data (e.g., Influenza), along with descriptive summaries of simple
spatio-temporal patterns and trends. However, public health researchers, government agencies,
as well as the general public, are often interested in deeper patterns and insights
into how the disease is spreading.
Analysis by domain experts is needed for deriving such insights from
incidence data.

\subsection*{Objective}
Our goal is to develop an automated approach for finding 
interesting spatio-temporal patterns in the spread of a disease
over a large region, such as:
regions which have specific characteristics, e.g., high incidence
in a particular week, those which showed a sudden change in incidence,
or regions which have significantly different incidence compared to
earlier seasons.

\subsection*{Methods}
We develop techniques from the area of transactional data mining for characterizing
and finding interesting spatio-temporal patterns in disease spread in an
automated manner.  A key part of our approach involves using the principle of
minimum description length for characterizing a set of regions in terms of
combinations of attributes; we use integer programming to find such descriptions.
Our automated approach explores regions that have different kinds of temporal
patterns, and ranks them based on their description length.

\subsection*{Results}
We apply our methods for finding spatio-temporal patterns in the spread of
seasonal Influenza in the US at the resolution of states. We find succinct
descriptions for regions (sets of states) with specific characteristics,
e.g., high activity level, which give better insight into such regions.
Our approach also finds interesting patterns in the form of
regions exhibiting significant changes in activity levels in a short time,
and in terms of activity levels in the past seasons.

\subsection*{Conclusions}
Our approach can provide new insights into the patterns and trends
in disease spread in an automated manner.
Our results show that the description complexity is an effective approach for
characterizing sets of interest, which can be easily extended to other
diseases and regions, beyond Influenza in the US.
The patterns we find have a specific
structure, which can be easily adapted for automated generation of narratives.
