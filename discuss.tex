\section*{Discussion}
Our results suggest that techniques from the area of transactional data mining 
are useful for finding spatio-temporal patterns in disease spread.
In particular, the minimum description length (MDL) principle is able to identify 
interesting patterns. 
We find that the relaxed versions, and allowing negative clauses
can significantly reduce the complexity of descriptions in many cases. 
Our ranking method also provides a systematic approach to identify trends and
surprises in the spread of ILI.
%Further, the structure of our representations, specifically,
%the positive and negative clauses, allows such representations to be converted into textual
%descriptions easily. 
However, the descriptions of high score are not always intuitive
or interesting. Instead, our ranking based approach (or other variations of it) could
help provide new insights to a domain expert, who might be able to find 
interesting spatio-temporal patterns more easily. Thus such an approach could be a first
step in processing epidemic incidence data.
We believe that including more characteristics for the data (i.e., more columns in the
data matrix $D$) can help find more succinct descriptions.
Further, the integer programming based approach is quite powerful, and more
constraints can be easily added to generate descriptions with specific kinds of properties.
Though the descriptions reported here were generated by hand,
these are all very well structured, and could conceivably be generated using
natural language processing techniques easily.

%%%pertaining to spread of Influenza. We also use the principle of minimum description length for characterizing a set of regions in terms of combination of attributes. To find such descriptions we use integer programming which find a description for the set of regions with the minimum description length possible from the combinations of attributes. As our results demonstrate, these description could shed light on many hidden patterns in the data such as trends and surprises. Also, we develop ideas to rank these patterns based on their interestingness. 

%%%This work develops techniques from the area of transactional data mining for finding spatio-temporal patterns pertaining to spread of Influenza. We also use the principle of minimum description length for characterizing a set of regions in terms of combination of attributes. To find such descriptions we use integer programming which find a description for the set of regions with the minimum description length possible from the combinations of attributes. As our results demonstrate, these description could shed light on many hidden patterns in the data such as trends and surprises. Also, we develop ideas to rank these patterns based on their interestingness. 

\subsection*{Limitations}

The feature values are real numbers, e.g., the similarity with a past season can be a correlation metric, not binary. One way to handle this issue would be to map the non-binary values to binary using discretization of the weights. Since we limited our focus to only meaningful features, our current approach explores target sets with temporal properties over small time intervals. In case of an increase in number of features by a few orders of magnitude than we considered, the ILP may not be able to scale well. One way to address this problem is to design scalable heuristics that give some theoretical/ experimental guarantees.

%Our current approach only explores target sets with temporal properties over small time intervals.Extending this to consider temporal properties over a long time is an interesting, but challenging problem. Also, the universe of elements could be the set of counties instead of states. This will add many more elements and features, thus increasing the problem size. The ILP discussed in our work may not be able to scale well in this case. One potential way to encounter this problem is to design scalable heuristics that solve this problem with some theoretical/ experimental guarantees.

%%Our ranking of interesting patterns is based on the assumption that the most interesting patterns are those with high variations in ILI activity levels in consecutive periods of time. This notion may not consider some other interesting patterns where the ILI activity have gradual decline or gradual increase in consecutive weeks.


\subsection*{Conclusion}
Automated generation of interesting spatio-temporal patterns and trends can be
very useful to public health experts, as well as the general public.
Our approach, based on techniques from pattern mining, can help provide a short-list
of patterns, which can then be examined more carefully by a domain expert.
The techniques developed in this paper could potentially be applied for other diseases,
and other public health domains. 

%%The automation of generating descriptive summaries of spatio-temporal patterns and trends could be useful to public health experts. Our methods tend to generate succinct descriptions to characterize regions with specific patterns of epidemic spread. Since domain experts used to manually generate such reports, it may not be easy for them to find interesting patterns that are latent in the data. Our approach provides the trends and surprises that are generated automatically and ranked as per their interestingness. Also, the techniques developed in this paper could potentially be applied in various other domains. Furthermore, the methods described here could be directly used to identify trends related to several other diseases. 